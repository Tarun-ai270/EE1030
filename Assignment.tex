i
%iffalse
\let\negmedspace\undefined
\let\negthickspace\undefined
\documentclass[journal,12pt,twocolumn]{IEEEtran}
\usepackage{cite}
\usepackage{amsmath,amssymb,amsfonts,amsthm}
\usepackage{algorithmic}
\usepackage{graphicx}
\usepackage{textcomp}
\usepackage{xcolor}
\usepackage{txfonts}
\usepackage{listings}
\usepackage{enumitem}
\usepackage{mathtools}
\usepackage{gensymb}
\usepackage{comment}
\usepackage[breaklinks=true]{hyperref}
\usepackage{tkz-euclide} 
\usepackage{listings}
\usepackage{gvv}                                        
%\def\inputGnumericTable{}                                 
\usepackage[latin1]{inputenc}                                
\usepackage{color}                                            
\usepackage{array}                                            
\usepackage{longtable}                                       
\usepackage{calc}                                             
\usepackage{multirow}                                         
\usepackage{hhline}                                           
\usepackage{ifthen}                                           
\usepackage{lscape}
\usepackage{tabularx}
\usepackage{array}
\usepackage{float}


\newtheorem{theorem}{Theorem}[section]
\newtheorem{problem}{Problem}
\newtheorem{proposition}{Proposition}[section]
\newtheorem{lemma}{Lemma}[section]
\newtheorem{corollary}[theorem]{Corollary}
\newtheorem{example}{Example}[section]
\newtheorem{definition}[problem]{Definition}
\newcommand{\BEQA}{\begin{eqnarray}}
\newcommand{\EEQA}{\end{eqnarray}}
\newcommand{\define}{\stackrel{\triangle}{=}}
\theoremstyle{remark}
\newtheorem{rem}{Remark}

% Marks the beginning of the document
\begin{document}
\bibliographystyle{IEEEtran}
\vspace{3cm}

\title{3.Quadratic Equation and\\ Inequations (Inequalities)}
\author{AI24BTECH11023 - Pakala Tarun Reddy}
\maketitle{Section-A JEE Advanced/IIT JEE}\\\\
\maketitle{E.Subjective Problems}\\\\

\begin{enumerate}[start=13]
\item Show that the equation $e^{\sin{x}}-e^{-\sin{x}}-4=0$ has no real solution.\hfill{(1982-2 Marks)}\\\\
\item mn squares of equal size are arranged to form a rectangle of dimensions $m$ by $n$, where $m$ and $n$ are natural numbers. Two squares will be called 'neighbours' if they have exactly one common side. A natural number is written in each square such that the number written in any square is the arithmetic mean of the numbers written in its neighbouring squares.Show that this is possible only if all the numbers used are equal. \hfill{(1982-5 Marks)}\\\\
\item If one root of the quadratic equation $ax^2+bx+c=0$ is equal to the $n$-the power of the other, then show that\\\\
$(ac^n)^\frac{1}{n+1} +(a^nc)^\frac{1}{n+1}+b=0$. \hspace*{\fill}\brak{1983-2 Marks}\\\\
 \item Find all real values of $x$ which satisfy $x^2-3x+2>0$ and $x^2-2x-4\leq 0$.\hspace*{\fill}\brak{1983-2 Marks}\\\\
 \item Solve for $x$; $(5+2\sqrt{6})^{x^2-3}+(5-2\sqrt{6})^{x^2-3}=10$.\\ \hspace*{\fill}\brak{1985-5 Marks}\\\\
 \item For $a \leq 0$, determine all real roots of the equation $x^2-2| x-a |-3a^2=0$. \hspace*{\fill}\brak{1986-5 Marks}\\\\
 \item Find the set of all $x$ for which $\frac{2x}{(2x^2+5X+2)}>\frac{1}{(x+1)}$ \hspace*{\fill}\brak{1987-3 Marks}\\\\
 \item Solve $| x^2+4x+3 |+2x+5=0$. \hspace*{\fill}\brak{1988-5 Marks}\\\\
 \item Let $a,b,c$ be real. If $ax^2+bx+c=0$ has two real roots $\alpha$ and $\beta$ , where $\alpha$ $<$ -1 and $\beta$ $>$ 1 , then show that $1+\frac{c}{a}+|\frac{b}{a}|<0$.
 \hfill{(1995- 5 Marks)}\\\\
 \item Let S be a square of unit area. Consider any quadrilateral which has one vertex on each side of S. If a, b, c, and d denote the lengths of the sides of the quadrilateral, prove that \\$2 \leq a^2+b^2+c^2+d^2 \leq 4$. \hfill{(1997-5 Marks)}\\\\
 
 \item If $\alpha$, $\beta$ are the roots of $ax^2+bx+c=0$, $(a\neq 0)$ and $\alpha$+$\delta$, $\beta$+$\delta$ are the roots of $Ax^2+Bx+C=0$, $(A\neq 0)$ for some constant $\delta$, then prove that $\frac{b^2-4ac}{a^2}=\frac{B^2-4AC}{A^2}$ \hfill{(2000-4 Marks)} \\\\

 \item Let $a, b,c $ be real numbers with $a\neq0$ and let $\alpha$ and $\beta$ be the roots of the equation $ax^2+bx+c=0$. Express the roots of $a^3x^2+abcx+c^3=0$ in terms of $\alpha$, $\beta$. \hspace*{\fill}\brak{2001-4 Marks}\\\\
\item If $x^2+(a-b)x+(1-a-b)=0$ where $a, b\in R$ then find the values of $a$ for which equation has unequal real roots for all values of $b$. \hspace*{\fill}\brak{2003-4 Marks}\\\\
\item If $a, b, c$ are positive real numbers. Then prove that $(a+1)^7(b+1)^7(c+1)^7 > 7^7a^4b^4c^4$. \hspace*{\fill}\brak{2004-4 Marks}\\\\
\item Let $a$ and $b$ be the roots of the equation $x^2-10cx-11d=0$ and those of $x^2-10ax-11b=0$ are $c,d $ then the value of $a=b+c+d$, when $a\neq b\neq c\neq d$, is. \hfill{(2006-6 Marks)}

 
 


\end{enumerate}
\bigskip

\renewcommand{\thefigure}{\theenumi}
\renewcommand{\thetable}{\theenumi}





\end{document}
