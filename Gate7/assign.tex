%iffalse
\let\negmedspace\undefined
\let\negthickspace\undefined
\documentclass[journal,12pt,onecolumn]{IEEEtran}
\usepackage{cite}
\usepackage{amsmath,amssymb,amsfonts,amsthm}
\usepackage{algorithmic}
\usepackage{graphicx}
\usepackage{textcomp}
\usepackage{xcolor}
\usepackage{txfonts}
\usepackage{listings}
\usepackage{enumitem}
\usepackage{mathtools}
\usepackage{pgfplots}
\usepackage{gensymb}
\usepackage{circuitikz}
\usepackage{comment}
\usepackage[breaklinks=true]{hyperref}
\usepackage{tkz-euclide} 
\usepackage{listings}
\usepackage{gvv}                                        
%\def\inputGnumericTable{}                                 
\usepackage[latin1]{inputenc}                                
\usepackage{color}                                            
\usepackage{array}                                            
\usepackage{longtable}                                       
\usepackage{calc}                                             
\usepackage{multirow}                                         
\usepackage{hhline}                                           
\usepackage{ifthen}                                           
\usepackage{lscape}
\usepackage{tabularx}
\usepackage{array}
\usepackage{float}

\usepackage{enumitem}
\usepackage{xcolor}
%\usepackage{multicol}


\newtheorem{theorem}{Theorem}[section]
\newtheorem{problem}{Problem}
\newtheorem{proposition}{Proposition}[section]
\newtheorem{lemma}{Lemma}[section]
\newtheorem{corollary}[theorem]{Corollary}
\newtheorem{example}{Example}[section]
\newtheorem{definition}[problem]{Definition}
\newcommand{\BEQA}{\begin{eqnarray}}
\newcommand{\EEQA}{\end{eqnarray}}
\newcommand{\define}{\stackrel{\triangle}{=}}
\theoremstyle{remark}
\newtheorem{rem}{Remark}

\title{2017-EE-1-13}
\author{AI24BTECH11023 - Tarun Reddy Pakala}
\begin{document}
\bibliographystyle{IEEEtran}

\maketitle
\bigskip
\renewcommand{\thefigure}{\theenumi}
\renewcommand{\thetable}{\theenumi}
\begin{enumerate}
\item The matrix \textbf{A}=$
\begin{bmatrix}
\frac{3}{2} & 0 & \frac{1}{2} \\
0 & -1 & 0 \\
\frac{1}{2} & 0 & \frac{3}{2}
\end{bmatrix}
$ has three distinct eigenvalues and one of its eigenvectors is $
\begin{bmatrix}
1  \\
0  \\
1 
\end{bmatrix}
$. Which one of the following can be another eigenvector of \textbf{A}?
\begin{enumerate}
    \item $
\begin{bmatrix}
0  \\
0  \\
-1 
\end{bmatrix}
$
    \item $
\begin{bmatrix}
-1  \\
0  \\
0
\end{bmatrix}
$
    \item $
\begin{bmatrix}
1 \\
0  \\
-1 
\end{bmatrix}
$
    \item $
\begin{bmatrix}
1  \\
-1  \\
1 
\end{bmatrix}
$
\end{enumerate}
\item For a complex number $z,\;\lim_{z\to i}\frac{z^2+1}{z^3+2z-1\brak{z^2+2}}$ is
\begin{enumerate}
    \item $-2i$
    \item $-i$
    \item $i$
    \item $2i$
\end{enumerate}
\item Let $z\brak{t}=x\brak{t}*y\brak{t}$ where "*" denotes convolution. Let $c$ be a positive real-valued constant. Choose the correct expression for $z\brak{ct}$. 
\begin{enumerate}
    \item $c\cdot x\brak{ct}*y\brak{ct}$
    \item $x\brak{ct}*y\brak{ct}$
    \item $c\cdot x\brak{t}*y\brak{ct}$
    \item $c\cdot x\brak{ct}*y\brak{t}$
\end{enumerate}
\item A solid iron cylinder is placed in a region containing a uniform magnetic field such that the cylinder axis is parallel to the magnetic field direction. The magnetic field lines inside the cylinder will
\begin{enumerate}
    \item bend closer to the cylinder 
    \item bend farther away from the axis
    \item remain uniform as before 
    \item cease to exist inside the cylinder
\end{enumerate}
\item Consider an electron, a neutron and a proton initially at rest and placed along a straight line such that the neutron is exactly at the center of the line joining the electron and proton. At $t=0$, the particles are released but are constrained to move along the same straight line. Which of these will collide first? 
\begin{enumerate}
    \item the particles will never collide
    \item all will collide together
    \item proton and neutron
    \item electron and neutron
\end{enumerate}
\item The transfer function of a system is given by. $$\frac{V_o\brak{s}}{V_i\brak{s}}=\frac{1-s}{1+s}$$ Let the output of the system be $v_o\brak{t}=V_m\sin\brak{\omega t+\phi}$ for the input, $v_i\brak{t}=V_m\sin \brak{\omega t}$. Then the minimum and maximum values of $\phi$ \brak{\text{in radius}} are respectively
\begin{enumerate}
    \item $\frac{-\pi}{2}$ and $\frac{\pi}{2}$
    \item $\frac{-\pi}{2}$ and $0$
    \item $0$ and $\frac{\pi}{2}$
    \item $-\pi$ and $0$
\end{enumerate}
\item Consider the system with following input-output relation $$y\sbrak{n}=\brak{1+\brak{-1}^n}x\sbrak{n}$$ where, $x\sbrak{n}$ is the input and $y\sbrak{n}$ is the output. The system is
\begin{enumerate}
    \item invertible and time invariant 
    \item invertible and time varying
    \item non-invertible and time invariant
    \item non-invertible and time varying
\end{enumerate}
\item A $4$ pole induction machine is working as an induction generator. The generator supply frequency is $60\;Hz$. The rotor current frequency is $5\;Hz$. The mechanical speed of the rotor in $RPM$ is
\begin{enumerate}
    \item $1350$
    \item $1650$
    \item $1950$
    \item $2250$
\end{enumerate}
\item A source is supplying a load through a 2-phase, 3-wire transmission system as shown in the figure below. The instantaneous voltage and current in phase are $v_{an}=220\sin{\brak{100\pi t}}\;V$ and $i_a=10\sin{\brak{100\pi t}}\;A$, respectively. Similarly for phase-b, the instantaneous voltage and current are $v_{bn}=220\cos{\brak{100\pi t}}\;V$ and $i_b=10\cos{\brak{100\pi t}}\;A$, respectively.
%input for figure 1
\begin{figure}[!ht]
    \centering
    \resizebox{3cm}{3cm}{%
        \begin{circuitikz}
            \tikzstyle{every node}=[font=\small]
            \draw [->, >=Stealth] (7,7.25) -- (7,12.5);
            \draw [->, >=Stealth] (6,9.75) -- (11,9.75);
            \draw [dashed] (7,8.25) -- (8.5,9.75);
            \draw [short] (8.5,9.75) -- (10,11.25);
            \draw [->, >=Stealth] (6.25,8.25) -- (6.75,8.25);
            \draw [->, >=Stealth] (8.5,10.5) -- (8.5,10);
            \node [font=\small] at (10.5,9.5) {T};
            \node [font=\small] at (6.5,11.25) {$\frac{1}{\chi}$};
            \node [font=\small] at (8.5,10.75) {600 K};
            \node [font=\small] at (6.75,9.5) {O};
            \node [font=\small] at (5.5,8.25) {-2 $\times 10^4$};
            \node [font=\small] at (5.5,8) {(CGS unit)};
        \end{circuitikz}
    }
    \label{fig:my_label}
\end{figure}

\begin{enumerate}
    \item $2200\;W$
    \item $2200\sin^2{\brak{100\pi t}}\;W$
    \item $4400\;W$
    \item $2200\sin{\brak{100\pi t}}\cos{\brak{100\pi t}}\;W$
\end{enumerate}
\item A 3-bus power system is shown in the figure below, where the diagonal element of $Y$-bus matrix are:$Y_{11}=-j12\;pu,\;Y_{22}=-j\;pu$ and $Y_{33}=-j7\;pu$.\\
%input for figure 2
\begin{figure}[H]
\centering
\resizebox{3cm}{3cm}{%
\begin{circuitikz}
\tikzstyle{every node}=[font=\small]
\draw  (3.5,12.5) rectangle (8,12.25);
\draw  (3.5,10.75) rectangle (8,10.5);
\draw [->, >=Stealth] (8.75,10.5) -- (8.75,11.25);
\draw [->, >=Stealth] (8.75,10.5) -- (9.5,10.5);
\draw [<->, >=Stealth] (4.25,10.75) -- (4.25,12.25);
\draw [short] (5.25,10.75) -- (5.25,12.25);
\draw [short] (5.25,10.75) -- (6.75,12.25);
\draw [->, >=Stealth] (5.25,11.25) -- (5.75,11.25);
\draw [->, >=Stealth] (5.25,12) -- (6.5,12);
\draw [->, >=Stealth] (5.25,11.5) -- (6,11.5);
\draw [->, >=Stealth] (5.25,11.75) -- (6.25,11.75);
\node [font=\small] at (4,11.5) {$d$};
\node [font=\small] at (5.25,12.75) {Moving plate};
\node [font=\small] at (5,10.25) {Fixed plate};
\node [font=\small] at (8.5,11.25) {$y$};
\node [font=\small] at (9.5,10.25) {$x$};
\node [font=\small] at (6.5,11.5) {$u\brak{y}$};
\end{circuitikz}
}%
\end{figure}

The per unit values of the line reactances $p,q$ and $r$ shown in the figure are
\begin{enumerate}
    \item $p=-0.2,\;q=-0.1,\;r=-0.5$
    \item $p=0.2,\;q=0.1,\;r=0.5$
    \item $p=-5,;q=-10,\;r=-2$
    \item $p=5,\;q=10,\;r=2$
\end{enumerate}
\item A closed loop system has the characteristic equation given by $s^3+Ks^2+\brak{K+2}s+3=0$. For this system to be stable, which one of the following conditions should be satisfied?
\begin{enumerate}
    \item $0\;<\;K\;<\;0.5$
    \item $0.5\;<\;K\;<\;1$
    \item $0\;<\;K\;<\;1$
    \item $K\;>\;1$
\end{enumerate}
\item The slope and level detector circuit in a $CRO$ has a delay of $100\;ns$. The start-stop sweep generator has a response time of $50\;ns$. In order to display correctly, a delay line of 
\begin{enumerate}
    \item $150\;ns$ has to be inserted into the $y$-channel
    \item $150\;ns$ has to be inserted into the $x$-channel
    \item $150\;ns$ has to be inserted into both $x$ and $y$ channels
    \item $100\;ns$ has to be inserted into both $x$ and $y$ channels
\end{enumerate}
\item The Boolean expression $AB+A\Bar{C}+BC$ simplifies to 
\begin{enumerate}
    \item $BC\;+A\Bar{C}$
    \item $AB\;+\;A\Bar{C}\;+\;B$
    \item $AB\;+\;A\Bar{C}$
    \item $AB\;+\;BC$
\end{enumerate}

\end{enumerate}
\end{document}
